\documentclass[10pt]{article}

\usepackage{mathtools, amssymb, bm}
\usepackage{microtype}
\usepackage[utf8]{inputenc}
\usepackage[top = 1.0in, left = 1.75in, right = 0.75in, bottom = 0.75in]{geometry}
\usepackage{booktabs}
\usepackage{graphicx}
\usepackage{xcolor}
\usepackage{tabularx}
\usepackage{tikzsymbols}
\usepackage[hidelinks]{hyperref}

\usepackage[explicit]{titlesec}
\titleformat{\section}[runin]{\bfseries}{}{0em}{
    \llap{
        \smash{
            \begin{tabularx}{0.75in}[t]{@{}l@{\hskip0.4em}>{\raggedright}X@{\hskip\marginparsep}}
                #1 
            \end{tabularx}
        }
    }
}[\leavevmode\hspace*{\dimexpr-\fontdimen2\font-\fontdimen3\font+0.25em}]

\usepackage{fancyhdr}
\pagestyle{fancy}
\fancyhf{}
\rhead{\thepage}
\renewcommand{\headrulewidth}{0pt}

\usepackage{lipsum}

%' ============================================================================================================================================================
%' ============================================================================================================================================================

\begin{document}

\newcommand{\mytitle}{Homework 2}
\newcommand{\myauthor}{Aiden Kenny}
\newcommand{\myclass}{STAT GR5203: Propbability}
\newcommand{\myschool}{Columbia University}
\newcommand{\mydate}{October 12, 2020}
\begin{flushright}
    \textbf{\mytitle}\\[0.5em]
    \textsl{\myauthor}\\
    \textsl{\myclass}\\
    \textsl{\myschool}\\
    \textsl{\mydate}
\end{flushright} \vspace{1em}

%' ============================================================================================================================================================
\section{Question 1} \noindent
Let \(X\) have a pdf of \(f(x) = c x^2\) for \(0 \le x \le 1\) and \(f(x) = 0\) elsewhere. 
\begin{itemize}
    \item[(a)] For this to be a valid pdf, it must integrate to \(1\) over the support. So 
    \begin{align*}
        \int_0^1 c x^2 \;\mathrm{d}x = \left. \frac{c x^3}{3} \right|_0^1 = \frac{c}{3} ~\overset{\text{set}}{=}~ 1,
    \end{align*}
    which leads to \(c = 3\). So \(f(x) = 3x^2\) for \(0 \le x \le 1\). 
    \item[(b)] The cdf is given by 
    \begin{align*}
        F(x) = \int_{-\infty}^x f(t) \;\mathrm{d}t = \int_0^x 3t^2 \;\mathrm{d}t = \left. t^3 \right|_0^x = x^3
    \end{align*}
    for \(0 \le x \le 1\). We also have \(F(x) = 0\) when \(x < 0\) and \(F(x) = 1\) when \(x > 1\).
    \item[(c)] We have 
    \begin{align*}
        \mathrm{Pr}\left(\frac{1}{10} \le X \le \frac{1}{2} \right) 
        = F\left( \frac{1}{2} \right) - F \left( \frac{1}{10} \right)
        = \frac{1}{2^3} - \frac{1}{10^3} = \frac{31}{250}.
    \end{align*}
\end{itemize}

%' ============================================================================================================================================================
\section{Question 2} \noindent
Two discrete random variables \(X\) and \(Y\) are jointly distributed. 
\begin{itemize}
    \item[(a)] The marginal pmf for \(X\) is obtained by summing over every value of \(Y\) for each value of \(X\).
    For example, to find the marginal probability that \(X=1\), we have \(f_X(x) = 0.10 + 0.05 + 0.02 + 0.02 = 0.19\). 
    The other values are obtained in the same way. Finding the marginal pmf for \(Y\) is done the exact same way, and 
    it turns out that \(f_X(j) = f_Y(j)\) for \(j = \{1,2,3,4\}\); they are both found in Table \ref{q02-tab}.
    \item[(b)] \(X\) and \(Y\) are not independent. For two random variables to be independent, we need \(f(x,y) = f_X(x)\cdot f_Y(y)\)
    for all possible \((x,y)\) pairs. Here we have \(f(1,1) = 0.10\) and \(f_X(1)\cdot f_Y(1) = 0.19^2 \neq f(1,1)\), meaing 
    \(X\) and \(Y\) are dependent. 
    \item[(c)] To find the conditional pmf of \(X\) given that \(Y=1\), we take each value of \(f(x,1)\) and divide by \(f_Y(1)\), 
    i.e. \(f_{X|Y}(x|1) = f(x,1) / f_Y(1)\). For example, we have \(f_{X|Y}(1|1) = 0.10 / 0.19 = 10/19\). Finding the conditional pmf for 
    \(Y\) given that \(X = 1\) is done in a similar way, and again they are the same. 
    Both pmfs can be found in be found in Table \ref{q02-tab}.
\end{itemize}
\begin{table}
    \centering
    \def\arraystretch{1.25}
    % \begin{tabular}[ht]{|c|cccc|} \hline
    %     \(_X\backslash^Y\) & 1 & 2 & 3 & 4 \\\hline
    %     1 & 0.10 & 0.05 & 0.02 & 0.02 \\
    %     2 & 0.05 & 0.20 & 0.05 & 0.02 \\
    %     3 & 0.02 & 0.05 & 0.20 & 0.04 \\
    %     4 & 0.02 & 0.02 & 0.04 & 0.10 \\\hline
    % \end{tabular}
    % \vspace{1em}

    \begin{tabular}[ht]{|c|cccc|} \hline
        \(x\) & 1 & 2 & 3 & 4 \\\hline
        \(f_X(x)\) & 0.19 & 0.32 & 0.31 & 0.18 \\
        \(f_{X|Y}(x|1)\) & 10/19 & 5/19 & 2/19 & 2/19 \\\hline
    \end{tabular}
    \quad
    \begin{tabular}[ht]{|c|cccc|} \hline
        \(y\) & 1 & 2 & 3 & 4 \\\hline
        \(f_Y(y)\) & 0.19 & 0.32 & 0.31 & 0.18 \\
        \(f_{Y|X}(y|1)\) & 10/19 & 5/19 & 2/19 & 2/19 \\\hline
    \end{tabular}
    \caption{Information for question 2.}
    \label{q02-tab}
\end{table}

%' ============================================================================================================================================================
\section{Question 3} \noindent
We are considering points \((x,y)\) uniformly selected within an ellipse given by the equation
% \begin{align*}
%     \left( \frac{x}{a} \right)^2 + \left( \frac{y}{b} \right)^2 = 1,
% \end{align*}
\((x/a)^2 + (y/b)^2 = 1\), where \(a,b > 0\). 
Therefore, the probability of selecting a point \((x,y)\) from this region is \(f(x,y) = c\) for 
% \(-a \le x \le a\) and \(-b\sqrt{1 - (x/a)^2} \le y \le b\sqrt{1 - (x/a)^2}\), and \(f(x,y) = 0\) elsewhere. 
\(-a\le x \le a\), \(-b \le y \le b\), and \((x/a)^2 + (y/b)^2 \le 1\), and \(f(x,y) = 0\) elsewhere. 
To find \(c\), we use the fact that a joint pdf must integrate to 1 over all values of \(x\) and \(y\), so 
\begin{align*}
    1 = \int_{-a}^{a} \int_{-b\sqrt{1 - (x/a)^2}}^{b\sqrt{1 - (x/a)^2}} c \;\mathrm{d}y \;\mathrm{d}x
    = c \cdot 2 \int_{-a}^a b\sqrt{1 - (x/a)^2} \;\mathrm{d}x
    = c \cdot \pi a b,
\end{align*}
which implies that \(c = 1 / \pi a b\). The last equality is from the fact that the area of an ellipse is \(\pi a b\), which is what the integral is computing. 
To find the marginal density of \(X\) (\(Y\)), we integrate over all possible values of \(Y\) (\(X\)):
\begin{align*}
    f_X(x) &= \int_{-b\sqrt{1 - (x/a)^2}}^{b\sqrt{1 - (x/a)^2}} \frac{1}{\pi a b} \;\mathrm{d}y
    = \frac{y}{\pi ab} \Big|_{-b\sqrt{1 - (x/a)^2}}^{b\sqrt{1 - (x/a)^2}} = \frac{2 \sqrt{1 - (x/a)^2}}{\pi a}
    ~~\text{for \(-a \le x \le a\)}, \\
    f_Y(y) &= \int_{-a\sqrt{1 - (y/b)^2}}^{a\sqrt{1 - (y/b)^2}} \frac{1}{\pi a b} \;\mathrm{d}x
    = \frac{x}{\pi ab} \Big|_{-a\sqrt{1 - (y/b)^2}}^{a\sqrt{1 - (y/b)^2}} = \frac{2 \sqrt{1 - (y/b)^2}}{\pi b}
    ~~\text{for \(-b \le y \le b\)}.
\end{align*}
These results make sense intuitively. For \(f_X(x)\), we can see that the highest probability is obtained when \(x = 0\). At \(x=0\), the 
height of the ellipse is the greatest, so there is more ``width'' for \(x = 0\) to be chosen. And as \(x \to \pm a\), \(f_X(x) \to 0\), meaning that the closer
you get to the end of the ellipse, the less likely that point is to be chosen. 



\end{document}