\documentclass[10pt]{article}

\usepackage{mathtools, amssymb, bm}
\usepackage{microtype}
\usepackage[utf8]{inputenc}
\usepackage[top = 1.0in, left = 1.75in, right = 0.75in, bottom = 0.75in]{geometry}
\usepackage{booktabs}
\usepackage{graphicx}
\usepackage{xcolor}
\usepackage{tabularx}
\usepackage{tikzsymbols}
\usepackage[hidelinks]{hyperref}

\usepackage[explicit]{titlesec}
\titleformat{\section}[runin]{\bfseries}{}{0em}{
    \llap{
        \smash{
            \begin{tabularx}{0.75in}[t]{@{}l@{\hskip0.4em}>{\raggedright}X@{\hskip\marginparsep}}
                #1 
            \end{tabularx}
        }
    }
}[\leavevmode\hspace*{\dimexpr-\fontdimen2\font-\fontdimen3\font+0.25em}]

\usepackage{fancyhdr}
\pagestyle{fancy}
\fancyhf{}
\rhead{\thepage}
\renewcommand{\headrulewidth}{0pt}

\usepackage{lipsum}

%' ============================================================================================================================================================
%' ============================================================================================================================================================

\begin{document}

\newcommand{\mytitle}{Homework 2}
\newcommand{\myauthor}{Aiden Kenny}
\newcommand{\myclass}{STAT GR5203: Propbability}
\newcommand{\myschool}{Columbia University}
\newcommand{\mydate}{October 12, 2020}
\begin{flushright}
    \textbf{\mytitle}\\[0.5em]
    \textsl{\myauthor}\\
    \textsl{\myclass}\\
    \textsl{\myschool}\\
    \textsl{\mydate}
\end{flushright} \vspace{1em}

%' ============================================================================================================================================================
\section{Question 1} \noindent
Let \(X\) have a pdf of \(f(x) = c x^2\) for \(0 \le x \le 1\) and \(f(x) = 0\) elsewhere. 
\begin{itemize}
    \item[(a)] For this to be a valid pdf, it must integrate to \(1\) over the support. So 
    \begin{align*}
        \int_0^1 c x^2 \;\mathrm{d}x = \left. \frac{c x^3}{3} \right|_0^1 = \frac{c}{3} ~\overset{\text{set}}{=}~ 1,
    \end{align*}
    which leads to \(c = 3\). So \(f(x) = 3x^2\) for \(0 \le x \le 1\). 
    \item[(b)] The cdf is given by 
    \begin{align*}
        F(x) = \int_{-\infty}^x f(t) \;\mathrm{d}t = \int_0^x 3t^2 \;\mathrm{d}t = \left. t^3 \right|_0^x = x^3
    \end{align*}
    for \(0 \le x \le 1\). We also have \(F(x) = 0\) when \(x < 0\) and \(F(x) = 1\) when \(x > 1\).
    \item[(c)] We have 
    \begin{align*}
        \mathrm{Pr}\left(\frac{1}{10} \le X \le \frac{1}{2} \right) 
        = F\left( \frac{1}{2} \right) - F \left( \frac{1}{10} \right)
        = \frac{1}{2^3} - \frac{1}{10^3} = \frac{31}{250}.
    \end{align*}
\end{itemize}

\end{document}