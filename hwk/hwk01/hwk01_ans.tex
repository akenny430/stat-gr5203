\documentclass[10pt]{article}

\usepackage{mathtools, amsfonts, bm}
\usepackage{microtype}
\usepackage[utf8]{inputenc}
\usepackage[top = 1.0in, left = 1.75in, right = 0.75in, bottom = 0.75in]{geometry}
\usepackage{booktabs}
\usepackage{graphicx}
\usepackage{xcolor}
\usepackage{tabularx}
\usepackage{tikzsymbols}
\usepackage[hidelinks]{hyperref}

\usepackage[explicit]{titlesec}
\titleformat{\section}[runin]{\bfseries}{}{0em}{
    \llap{
        \smash{
            \begin{tabularx}{0.85in}{r}
                #1 
            \end{tabularx}
        }
    }
}[\leavevmode\hspace*{\dimexpr-\fontdimen2\font-\fontdimen3\font}]

\usepackage{fancyhdr}
\pagestyle{fancy}
\fancyhf{}
\rhead{\thepage}
\renewcommand{\headrulewidth}{0pt}

\usepackage{lipsum}

%' ============================================================================================================================================================
%' ============================================================================================================================================================

\begin{document}

\newcommand{\mytitle}{Homework 1}
\newcommand{\myauthor}{Aiden Kenny}
\newcommand{\myclass}{STAT GR5203: Probability}
\newcommand{\myschool}{Columbia University}
\newcommand{\mydate}{September 29, 2020}
\begin{flushright}
    \textbf{\mytitle}\\[0.5em]
    \textsl{\myauthor}\\
    \textsl{\myclass}\\
    \textsl{\myschool}\\
    \textsl{\mydate}
\end{flushright} \vspace{1em}

%' ============================================================================================================================================================
\section{Question 1} \noindent
We will let H denote a heads and T denote a tails.
\begin{itemize}
    \item[(a)] The sample space is given by \(\mathcal{S} = \{\mathrm{HHH, HHT, HTH, HTT, THH, THT, TTH, TTT}\}\).
    \item[(b)] We have
    \begin{enumerate}
        \item \(A = \text{at least two heads} = \{\mathrm{HHH, HHT, HTH, THH}\}\).
        \item \(B = \text{the first two tosses are heads} = \{\mathrm{HHH, HHT}\}\).
        \item \(C = \text{the last toss is a tail} = \{\mathrm{HHT, HTT, THT, TTT}\}\).
    \end{enumerate}
    \item[(c)] We have 
    \begin{enumerate}
        \item \(A^c = \{\mathrm{HTT, THT, TTH, TTT}\}\).
        \item \(A \cap B = \{\mathrm{HHH, HHT}\} = B\) (since \(B \subset A\)).
        \item \(A \cup C = \{\mathrm{HHH, HHT, HTH, HTT, THH, THT, TTT}\}\).
    \end{enumerate}
\end{itemize}

%' ============================================================================================================================================================
\section{Question 2} \noindent


\end{document}