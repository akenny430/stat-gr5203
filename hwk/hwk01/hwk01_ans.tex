\documentclass[10pt]{article}

\usepackage{mathtools, amsfonts, bm}
\usepackage{microtype}
\usepackage[utf8]{inputenc}
\usepackage[top = 1.0in, left = 1.75in, right = 0.75in, bottom = 0.75in]{geometry}
\usepackage{booktabs}
\usepackage{graphicx}
\usepackage{xcolor}
\usepackage{tabularx}
\usepackage{tikzsymbols}
\usepackage[hidelinks]{hyperref}

\usepackage[explicit]{titlesec}
\titleformat{\section}[runin]{\bfseries}{}{0em}{
    \llap{
        \smash{
            \begin{tabularx}{0.85in}{r}
                #1 
            \end{tabularx}
        }
    }
}[\leavevmode\hspace*{\dimexpr-\fontdimen2\font-\fontdimen3\font+0.25em}]

\usepackage{fancyhdr}
\pagestyle{fancy}
\fancyhf{}
\rhead{\thepage}
\renewcommand{\headrulewidth}{0pt}

\usepackage{lipsum}

%' ============================================================================================================================================================
%' ============================================================================================================================================================

\begin{document}

\newcommand{\mytitle}{Homework 1}
\newcommand{\myauthor}{Aiden Kenny}
\newcommand{\myclass}{STAT GR5203: Probability}
\newcommand{\myschool}{Columbia University}
\newcommand{\mydate}{September 29, 2020}
\begin{flushright}
    \textbf{\mytitle}\\[0.5em]
    \textsl{\myauthor}\\
    \textsl{\myclass}\\
    \textsl{\myschool}\\
    \textsl{\mydate}
\end{flushright} \vspace{1em}

%' ============================================================================================================================================================
\section{Question 1} \noindent
We will let H denote a heads and T denote a tails.
\begin{itemize}
    \item[(a)] The sample space is given by \(\mathcal{S} = \{\mathrm{HHH, HHT, HTH, HTT, THH, THT, TTH, TTT}\}\).
    \item[(b)] We have
    \begin{enumerate}
        \item \(A = \text{at least two heads} = \{\mathrm{HHH, HHT, HTH, THH}\}\).
        \item \(B = \text{the first two tosses are heads} = \{\mathrm{HHH, HHT}\}\).
        \item \(C = \text{the last toss is a tail} = \{\mathrm{HHT, HTT, THT, TTT}\}\).
    \end{enumerate}
    \item[(c)] We have 
    \begin{enumerate}
        \item \(A^c = \{\mathrm{HTT, THT, TTH, TTT}\}\).
        \item \(A \cap B = \{\mathrm{HHH, HHT}\} = B\) (since \(B \subset A\)).
        \item \(A \cup C = \{\mathrm{HHH, HHT, HTH, HTT, THH, THT, TTT}\}\).
    \end{enumerate}
\end{itemize}

%' ============================================================================================================================================================
\section{Question 2} \noindent
For each of the scenarios, we are pulling five cards from a well-shuffled deck without replacement. Since the order of the cards does 
not matter, there are \(\binom{52}{5}\) possible combinations. 
% Also note that we are \textit{excluding} any better hands when counting
% the probabilities; for example, when finding the probability of a straight flush, we will not consider a royal flush.
\begin{itemize}
    \item[(a)] \textsl{Royal flush}: since the order does not matter, for a given suite there is only one possible royal flush, so there are 
    only four possible royal flushes (one for each suite). So the probability of getting a royal flush is \(4 / \binom{52}{5}\).
    \item[(b)] \textsl{Straight flush}: for a given suite, not counting the royal flush, there are nine possible card combinations that fit 
    this criteria, meaning there are 36 possible straight flushes. So the probability of getting a straight flush is \(36 / \binom{52}{5}\).
    \item[(c)] \textsl{Four of a kind}: to select four cards of the same value, we must select all four suites of a given value, of which there
    are 13. Once this happens, four of the five cards in the hand have been determined, and we just have to select the last card. There are 12 
    possible card values and four possible suites. Therefore, there are \(13\cdot12\cdot4 = 624\) plausible hands, and the probability of getting 
    a four of a kind is \(624 / \binom{52}{5}\).
    \item[(d)] \textsl{Flush}: each suite has 13 unique values, so there are \(\binom{13}{5}\) ways to select five cards for a given suite. 
    However, this is including the ten possible \textit{consecutive} hands, which must be removed (as that hand would be either a straigh or 
    royal flush). This can be done for each suite, so there are \(4 \left( \binom{13}{5} - 10 \right)\) plausible hands, and so the probability 
    of getting a flush is \(4 \left( \binom{13}{5} - 10 \right) / \binom{52}{5}\).
    \item[(e)] \textsl{Three of a kind}: to get three cards of the same value, for a given value we have to choose three cards from the four possible
    suites. There are \(\binom{4}{3}\) ways to do this for each of the 13 values. For the remaining two cards to be chosen, there are 12 possible 
    values to choose from (choosing the same value of the three matching cards would give us a four of a kind), and each of these two cards can be 
    any of the four suites. That is, there are \(\binom{12}{2}\cdot4\cdot4\) ways to choose the last two cards. Therefore, there are 
    \(\binom{4}{3}\cdot13\cdot\binom{12}{2}\cdot4\cdot4 = 208\binom{4}{3}\binom{12}{2}\) plausible hands, and so the probability of 
    getting a three of a kind is \(208\binom{4}{3}\binom{12}{2} / \binom{52}{5}\).
    \item[(f)] \textsl{Two pairs}: to get two pairs of cards of the same value, we have to have two unique values to begin with, and there are 
    \(\binom{13}{2}\) ways to choose them. For each value, we are choosing two of the four possible suites, so there are \(\binom{4}{2}\) choices 
    \textit{for each value}. For the last card, there are 11 possible values (choosing either of the previous values would result in a three of a kind),
    and from these there are four possible suites, so there are 44 ways to choose the last card. So there are \(44\binom{13}{2}\binom{4}{2}^2\) plausible 
    hands, and so the probability of getting two pairs is \(44\binom{13}{2}\binom{4}{2}^2 / \binom{52}{5}\).
\end{itemize}

%' ============================================================================================================================================================
\section{Question 3} \noindent
Let \(E\) be the event that the president is a woman, \(F\) be the event that the vice-president is a man, and \(G\) be the event that both leaders are of 
the same sex. Since we are choosing leaders without replacement and with regard to order, there are \(48\cdot47\) possible leadership arrangements. We are 
also assuming committee choices are independent. 
\begin{itemize}
    \item[(a)] For \(E\) to happen we only care that the president is a woman, the sex of the vice-president is irrelevant. So there are 16 possible choices 
    for the president, and then any of the 47 remianing members can be chosen. So there are \(16\cdot47\) outcomes in \(E\), and 
    \(\mathrm{Pr}(E) = \frac{16\cdot47}{48\cdot47} = 1/3\). Similarly, for \(F\) we only care about the gender of the vice president. Here there are two
    possibilities: a man is chosen as the president or a woman is chosen as the president. For the former, there are \(32\cdot 31\) possibilities, and for the 
    latter, there are \(16\cdot32\) possibilities. Since the two situations are disjoint, we have \(\mathrm{Pr}(F) = \frac{32\cdot31+16\cdot32}{48\cdot47} 
    = 2/3\). Finally, there are \(32\cdot31\) combinations of two male leaders and \(16\cdot15\) combinations of female leaders, and since the two 
    possibilities are disjoint, we have \(\mathrm{Pr}(G) = \frac{32\cdot31+16\cdot15}{48\cdot47} = 77/141\).
    \item[(b)] \(E\cap F\) is the event that the president is female and the vice president is male. There are \(16\cdot32\) ways this can happen, so 
    \(\mathrm{Pr}(E\cap F) = \frac{16\cdot32}{48\cdot47} = 32/141\). To find \(\mathrm{Pr}(E\cup F)\) (the probability that the president is female or 
    the vice-president is a male), we have \[\mathrm{Pr}(E \cup F) - \mathrm{Pr}(E) + \mathrm{Pr}(F) - \mathrm{Pr}(E\cap F) = 1/3 + 2/3 - 32/141 = 109/141.\]
    Finally, since the event \(E\cap F\) requires the leaders to be of opposite sex, we know that \((E\cap F)\cap G = \emptyset\), which means 
    \(\mathrm{Pr}(E \cap F \cap G) = 0\). 
    \item[(c)] There are two possibilities when the event \(G\) occurs: both leaders are male, or both leaders are female. Since these possibilities are 
    disjoint, we have \(G = (G \cap E) \cup (G \cap F)\). Then, using the definition of conditional probability, we have
    \begin{align*}
        \mathrm{Pr}(G | E \cup F) = \frac{\mathrm{Pr}(G \cap (E \cup F))}{\mathrm{Pr}(E \cup F)} 
        = \frac{\mathrm{Pr}\big( (G \cap E) \cup (G \cap F) \big)}{\mathrm{Pr}(E \cup F)} 
        = \frac{\mathrm{Pr}(G)}{\mathrm{Pr}(E \cup F)} = \frac{77}{109}.
    \end{align*}
\end{itemize}

%' ============================================================================================================================================================
\section{Question 4} \noindent
Since we want to consider the location of the four aces in the card deck, order does matter here, and so there are \(52!\) ways of shuffling the deck. 
% To answer this question, we will use a geometric argument
To solve this problem, we will first shuffle the four aces separately, then shuffle the remaining cards, then finally insert the four aces together into the
rest of the deck. There are \(4!\) ways to arrange the four aces and \(48!\) ways of arranging the remaining cards, and there are 49 possible spaces 
where the four aces can be inserted into the rest of the deck. Therefore, there are \(4!\cdot48!\cdot49 = 4!\cdot49!\) plausible deck arrangements, and so the probability
that the four aces are next to each other is \(4!\cdot49! / 52! = 1 / 5525\). 

%' ============================================================================================================================================================
\section{Question 5} \noindent
For this question, we only care about how the first 30 students are chosen, since that will automatically tell us the students in both classrooms, and there 
are \(\binom{60}{30}\) ways to choose 30 of the 60 students. 
\begin{itemize}
    \item[(a)] There is only 1 way to choose all five friends to be in the same class, and then there are \(\binom{55}{25}\) ways to choose the remaining 
    25 studnets. So the probability of all five friends being in the same class is \(\binom{55}{25} / \binom{60}{30}\).
    \item[(b)] There are \(\binom{5}{4}\) ways of choosing four of the five friends to be in the same class, and then \(\binom{56}{26}\) ways to choose the 
    remaining 26 students. So the probability that four of the five friends is in the same class is \(\binom{5}{4}\binom{56}{26} / \binom{60}{30}\). We can 
    see a more general pattern here: there are \(\binom{5}{j} \binom{60 - j}{30 - j}\) ways to have \(j\) of the five friends be in the same class, for 
    \(j = 0, \ldots,5\); in part (a), \(j = 5\), and in part (b), \(j=4\). 
    \item[(c)] While there are \(\binom{5}{4}\) ways for \textit{any} of the four friends to be chosen in the class, there is only \textit{one} way in which 
    Marcelle is left out. Since there are \(\binom{56}{26}\) ways to choose the other 26 students whenever Marcell is left out, the probability that Marcell 
    is in class by herself is \(\binom{56}{25} / \binom{60}{30}.\)
\end{itemize}

%' ============================================================================================================================================================
\section{Question 6} \noindent
Let \(I_j\) be the event that the stock's price moves up on day \(j\), and let \(D_j\) be the event that the stock's price moves down on day \(j\). 
We have \(\mathrm{Pr}(I_j) = p\) and \(\mathrm{Pr}(D_j) = 1 - p\) for all \(j\), the changes on each day are independent, and each unique sequence of 
events is disjoint. 
\begin{itemize}
    \item[(a)] Let \(E\) be the event that the stock's price stays the same after two days. There are two ways this can happen: \(I_1 \cap D_2\) or 
    \(D_1 \cap I_2\). So 
    \begin{align*}
        \mathrm{Pr}(E) &= \mathrm{Pr} \Big( (I_1 \cap D_2) \cup (D_1 \cap I_2) \Big) \\
        &= \mathrm{Pr}(I_1)\mathrm{Pr}(D_2) + \mathrm{Pr}(D_1) \mathrm{Pr}(I_2) 
        = p(1-p) + (1-p)p = 2p(1-p).
    \end{align*}
    \item[(b)] Let \(F\) be the event that the stock's price increases by 1 unit after three days. There are three ways this can happen: \(I_1 \cap D_2 \cap I_3\), 
    \(D_1 \cap I_2 \cap I_3\), or \(I_1 \cap I_2 \cap D_3\). So 
    \begin{align*}
        \mathrm{Pr}(F) &= \mathrm{Pr} \Big( (I_1 \cap D_2 \cap I_3) \cup (D_1 \cap I_2 \cap I_3) \cup (I_1 \cap I_2 \cap D_3) \Big) \\
        &= \mathrm{Pr}(I_1)\mathrm{Pr}(D_2)\mathrm{Pr}(I_3) + \mathrm{Pr}(D_1)\mathrm{Pr}(I_2)\mathrm{Pr}(I_3) + \mathrm{Pr}(I_1)\mathrm{Pr}(I_2)\mathrm{Pr}(D_3) \\
        &= p(1-p)p + (1-p)p^2 + p^2 (1-p) = 3p^2(1-p).
    \end{align*}
    \item[(c)] %We want to determine \(\mathrm{Pr}(I_1 | F)\). 
    We observe that of the three possibile scenarios in \(F\), two of them begin with \(I_1\), and so \(\mathrm{Pr}(I_1 \cap F) = 2 p^2 (1-p)\) (to be
    more rigorous, we could show that \(I_1 \cap F = 
    (I_1 \cap D_2 \cap I_3) \cup (I_1 \cap I_2 \cap D_3)\) and then calculate the probability directly). We then have 
    \begin{align*}
        \mathrm{Pr}(I_1 | F) = \frac{\mathrm{Pr}(I_1 \cap F)}{\mathrm{Pr}(F)} = \frac{2 p^2 (1-p)}{3 p^2 (1-p)} = \frac{2}{3}.
    \end{align*}
\end{itemize}

%' ============================================================================================================================================================
\section{Question 7} \noindent
Let \(C\) be the event that the question is answered correctly, and let \(I\) be the event that the question is answered incorrectly. 
Both the husband and the wife answer the question correctly with probability \(p\) and incorrectly with probability \(1-p\). 
\begin{itemize}
    \item[(a)] For the first strategy, let \(H\) be the event that the husband is chosen, and let \(W\) be the event that the wife is 
    chosen. We have \(\mathrm{Pr}(C | H) = \mathrm{Pr}(C | W) = p\), and since 
    we are randomly choosing either the husband or wife, so \(\mathrm{Pr}(H) = \mathrm{Pr}(W) = 1/2\). By the law of total 
    probability, we have \(\mathrm{Pr}(C) = \mathrm{Pr}(C | H) \mathrm{Pr}(H) + \mathrm{Pr}(C | W) \mathrm{Pr}(W) = p/2 + p/2 = p\).
    \item[(b)] For the second strategy, there are three possible ways that they can be correct:
    \begin{enumerate}
        \item Both the husband and wife are correct and agree.
        \item The husband is correct and the wife is incorrect, so they flip a coin and are correct.
        \item The wife is correct and the husband is incorrect, so they flip a coin and are correct.
    \end{enumerate}
    The probability of the first scenario is \(p^2\). In both the second and third scenarios, the probability of them 
    disagreeing is \(p(1-p)\). When flipping a coin, the probability that the coin is correct is \(1/2\), so the probability
    of the couple disagreeing and the coin being correct is \(p(1-p)/2\), since the two events are independent. Finally, since all three
    scenarios are disjoint, we have 
    \begin{align*}
        \mathrm{Pr}(C) = p^2 + \frac{p(1-p)}{2} + \frac{p(1-p)}{2} = p^2 + p - p^2 = p.
    \end{align*}
\end{itemize}
We have shown that \textit{the probability of being correct is the same for both stragies}. 

%' ============================================================================================================================================================
\section{Question 8} \noindent
Let \(p = 0.6\), let \(A\) be the event that the couple agrees, and let \(D\) be the event that the couple disagrees. We note that \(A \cap D = \emptyset\), 
since the couple cannot agree and disagree at the same time. 
\begin{itemize}
    \item[(a)] We note that \(\mathrm{Pr}(A \cap C) = p^2\) and \(\mathrm{Pr}(A \cap I) = (1-p)^2\). Then using Bayes' theorem, we have 
    \begin{align*}
        \mathrm{Pr}(C | A) = \frac{\mathrm{Pr}(A \cap C)}{\mathrm{Pr}(A)} = \frac{\mathrm{Pr}(A \cap C)}{\mathrm{Pr}(A \cap C) + \mathrm{Pr}(A \cap I)} 
        = \frac{p^2}{p^2 + (1-p)^2}.
    \end{align*}
    When \(p = 0.6\), we have \(\mathrm{Pr}(C | A) = 0.6923077\).
    \item[(b)] When the couple disagrees, their answer being correct comes down to the coin flip, and so \(\mathrm{Pr}(C | D) = 1/2\). 
\end{itemize}
So, using the second strategy, the couple is more likely to be correct if they agree on the answer. 

%' ============================================================================================================================================================
\section{Question 9} \noindent
There are \(n\) independent coint flips, and for each coin toss we have \(\mathrm{Pr}(H) = p\). Let \(K\) denote the number of heads that show up throughout
the \(n\) flips. We want to determine how large \(n\) should be such that \(\mathrm{Pr}(K \ge 1) \ge 1/2\). First, we observe that \(\mathrm{Pr}(K \ge 1) = 
1 - \mathrm{Pr}(K = 0)\). Since \(K = 0\) corresponds to observing \(n\) tails, we have \(\mathrm{Pr}(K = 0) = (1-p)^n\), and so we have 
\(1 - (1-p)^n \ge 1/2\). Slightly rearranging gives us \((1-p)^n \le 1/2\), and taking the log of both sides gives us \(n \ln(1 - p) \le \ln(1/2)\). Since 
\(1 - p \le 1\), we have \(\ln(1-p) \le 0\), and dividing by it will flip the inequality. Therefore, we need 
\begin{align*}
    n \ge \frac{\ln(1/2)}{\ln(1-p)} = - \frac{\ln 2}{\ln (1-p)}.
\end{align*}
We note that as \(p \to 1\), this lower bound approaches zero, and as \(p \to 0\), the lower bound approaches infinity; 
both of these results make sense intuitively. 

%' ============================================================================================================================================================
\section{Question 10} \noindent
Let \(S_1\) be the event that the first coin is silver, and let \(S_2\) be the event that the second coin is silver. Each of the three cabinets has an equal
chance of being selected, so \(\mathrm{Pr}(A) = \mathrm{Pr}(B) = \mathrm{Pr}(C) = 1/3\). Since the drawers are opened at random as well, we have 
\(\mathrm{Pr}(S_1 | A) = 0\), \(\mathrm{Pr}(S_1 | B) = 1\), and \(\mathrm{Pr}(S_1 | C) = 1/2\). We also note that since cabinet \(B\) is the only cabinet to 
have two silver coins, we have \(\mathrm{Pr}(S_2 \cap S_1) = \mathrm{Pr}(B) = 1/3\). Using Bayes' theorem, we have 
\begin{align*}
    \mathrm{Pr}(S_2 | S_1) &= \frac{\mathrm{Pr}(S_2 \cap S_1)}{\mathrm{Pr(S_1)}} = \frac{\mathrm{Pr}(B)}{\mathrm{Pr}(S_1|A)\mathrm{Pr}(A) +
    \mathrm{Pr}(S_1|B)\mathrm{Pr}(B) + \mathrm{Pr}(S_1|C)\mathrm{Pr}(C)} \\
    &= \frac{1/3}{\left( 1 + 0 + 1/2 \right)/3} = \frac{1}{3/2} = \frac{2}{3}.
\end{align*}



\end{document}