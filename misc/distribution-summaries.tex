% \documentclass[9pt]{beamer}
\documentclass[8pt, handout]{beamer}
% \usepackage{pgfpages}
% \pgfpagesuselayout{4 on 1}[border shrink = 5mm]

%' ============================================================================================================================================================
% FOOTER
% \def\mysep{\(\diamond\)}
\def\mysep{\(\circ\)}
\makeatletter
\setbeamertemplate{footline}
{
    \dimen0 = \paperwidth
    \multiply\dimen0 by \insertframenumber
    \divide\dimen0 by \inserttotalframenumber
    \edef\progressbarwidth{\the\dimen0}
    
    {\leavevmode%
    \hbox{%
    \begin{beamercolorbox}[wd = 0.805\paperwidth, ht = 4ex , dp = 2ex, left]{section in head/foot}%
        % ~~~~\insertshorttitle~\insertheadsection~\insertheadsubsection
        \ifx\inserttitle\empty~~~~
        \else~~~~\inserttitle~~\mysep~
        \fi
        %
        \ifx\insertsubtitle\empty 
        \else\insertsubtitle~~\mysep~
        \fi
        %
        \ifx\insertauthor\empty
        \else\insertauthor~~\mysep~
        \fi
        %
        \ifx\insertdate\empty
        \else\insertdate
        \fi
    \end{beamercolorbox}%
    \begin{beamercolorbox}[wd = 0.195\paperwidth, ht = 4ex, dp = 2ex, right]{section in head/foot}%
        \hspace{0.25in}\insertframenumber ~/ \inserttotalframenumber~~~
    \end{beamercolorbox}%
    }%
    \vskip0pt%
    }%
    \begin{beamercolorbox}[wd = \paperwidth, ht = 0.6ex, dp = 0ex]{author in head/foot}%
        \begin{beamercolorbox}[wd = \progressbarwidth, ht = 0.6ex, dp = 0ex]{title in head/foot}%
        \end{beamercolorbox}%
    \end{beamercolorbox}%
}
\makeatother

%' ============================================================================================================================================================
% TITLE
\usepackage{fontawesome}
\newcommand{\setFA}[1]{\begingroup\fontseries{m}\fontshape{n}\selectfont{#1}\endgroup}

\newcommand{\email}[1]{\href{#1}{\setFA{\faEnvelopeO}}}
\newcommand{\website}[1]{\href{#1}{\setFA{\faDesktop}}}
\newcommand{\github}[1]{\href{#1}{\setFA{\faGithub}}}
\newcommand{\twitter}[1]{\href{#1}{\setFA{\faTwitter}}}
\newcommand{\linkedin}[1]{\href{#1}{\setFA{\faLinkedin}}}
\newcommand{\youtube}[1]{\href{#1}{\setFA{\faYoutube}}}
\defbeamertemplate*{title page}{customized}[1][]
{
    \centering
    \vspace{0.5in}
    \ifx\insertsubtitle\empty {\LARGE\inserttitle} \par \vspace{0.5in}
    \else {\LARGE\inserttitle} \par \vspace{0.1in} {\insertsubtitle} \par \vspace{0.5in}
    \fi
    %
    {\insertauthor} \par \vspace{0.05in}
    {\insertinstitute} \par \vspace{0.05in}
    {\insertdate} \par \vspace{0.5in}
    %
    {
        \email{mailto:apk2152@columbia.edu}~~
        \linkedin{https://linkedin.com/in/akenny430}~~
        \github{https://github.com/akenny430}
    } \par
}

%' ============================================================================================================================================================
% SECTIONS
\AtBeginSection[]{
    \begin{frame}{}
    
    \centering
    \vfill
    \vspace{0.5in}
    \LARGE\insertsection
    \vfill
    
    \end{frame}
}

%' ============================================================================================================================================================
% GENERAL
% \hypersetup{colorlinks = false}
\setbeamersize{text margin left = 0.25in, text margin right = 0.25in}
\setbeamertemplate{navigation symbols}{}
\setbeamertemplate{itemize item}[circle]{}
\setbeamertemplate{itemize subitem}[square]{}
\setbeamercolor{itemize item}{fg = textcol}
\setbeamercolor{itemize subitem}{fg = textcol}
\setbeamertemplate{enumerate item}[default]{}
\setbeamercolor{enumerate item}{fg = textcol}
\setbeamertemplate{itemize/enumerate subbody begin}{\normalsize}

% COLORS
\setbeamercolor{frametitle}{bg = bgcol, fg = textcol}                   % for slide title
\setbeamercolor{title in head/foot}{bg = maincol, fg = textcol}         % for filled progress bar
\setbeamercolor{author in head/foot}{bg = maincol!50, fg = textcol}     % for unfilled progress bar
\setbeamercolor{section in head/foot}{bg = bgcol, fg = textcol}         % for stuff in footer
\setbeamercolor{background canvas}{bg = bgcol}                          % for slide background
\setbeamercolor{normal text}{fg = textcol}                              % for slide text color

%' ============================================================================================================================================================
%' ============================================================================================================================================================

% packages used to adjust formatting of document
\usepackage{microtype}
\usepackage[light]{roboto}
\usefonttheme[onlymath]{serif}
\urlstyle{same}

% making tables
\usepackage{booktabs}
\usepackage{multirow, multicol}
\usepackage{tabularx}

% math
\usepackage{mathtools, amssymb, bm} 

% images
\usepackage{graphicx}
\usepackage{xcolor}

% extra
\usepackage{tikzsymbols}
    
% my custom definitions
\renewcommand{\emph}[1]{\textcolor{defcol}{\textsl{#1}}}

%' ============================================================================================================================================================
%' ============================================================================================================================================================
%' 
%' ALL THE STUFF ABOVE IS MODIFYING THE 
%' 
%' ============================================================================================================================================================
%' ============================================================================================================================================================







\title{Summary of Probability Distributions}
% \subtitle{}
\author{Aiden Kenny}
\institute{STAT GR5203: Probability}
\date{October 15, 2020}

% main color
% \definecolor{maincol}{HTML}{009E73} % green
% \definecolor{maincol}{HTML}{FF5833} % red-orange
\definecolor{maincol}{HTML}{0071ce} % blue

% color themes
\definecolor{bgcol}{HTML}{ffffff} \definecolor{textcol}{HTML}{000000} \definecolor{defcol}{HTML}{960018} % white
% \definecolor{bgcol}{HTML}{f8f8f8} \definecolor{textcol}{HTML}{000000} \definecolor{defcol}{HTML}{960018} % light
% \definecolor{bgcol}{HTML}{2c2c2e} \definecolor{textcol}{HTML}{f2f2f7} \definecolor{defcol}{HTML}{f0e442} % dark

\begin{document}

%' ============================================================================================================================================================
\begin{frame}{}

    \maketitle

\end{frame}

%' ============================================================================================================================================================
\begin{frame}{Random variables}

    \begin{itemize}
        \item Suppose you have an experiment with sample space \(\mathcal{S}\).
        \item A \emph{random variable} is a function \(X : \mathcal{S} \to \mathbb{R}\).
        \item Assigns a numerical value to all outcomes of experiment. 
        % \begin{itemize}
        %     \item e.g. 
        % \end{itemize}
        \vspace{2em}
        \item The \emph{support} of \(X\) is defined as \(\mathcal{D} = \mathrm{img}(X) \subseteq \mathbb{R}\).
        \item The support is all possible values that \(X\) can obtain. 
        \vspace{2em}
        % \item Let \(\mathcal{C} \subseteq \mathcal{D}\) be an event. 
        \item A random variable is \emph{discrete} if \(\mathrm{card}(\mathcal{D}) \le \mathrm{card}(\mathbb{N})\).
        \item When \(\mathrm{card}(\mathcal{D}) < \mathrm{card}(\mathbb{N})\), \(X\) has finitely many values.
        \item When \(\mathrm{card}(\mathcal{D}) = \mathrm{card}(\mathbb{N})\), \(X\) has countably infinite values. 
        \vspace{2em}
        \item A random variable is \emph{continuous} if \(\mathrm{card}(\mathcal{D}) > \mathrm{card}(\mathbb{N})\).
    \end{itemize}

\end{frame}

\section{Discrete Distributions}

%' ============================================================================================================================================================
\begin{frame}{Useful summations}

    \emph{Geometric series:} for all \(x \in \mathbb{R}\) and \(|r| < 1\), we have
    \begin{align*}
        \frac{x(1 - r^{n+1})}{1 - r} &= \sum_{k=0}^{n} x r^k \\%= \sum_{k=1}^{n} x r^{k-1} \\
        \frac{x}{1 - r} &= \sum_{k=0}^{\infty} x r^k %= \sum_{k=1}^{\infty} x r^{k-1}.
    \end{align*}

    % \vspace{2em}
    \emph{Binomial series:} for all \(x,y\in\mathbb{R}\) and \(n \in \mathbb{Z}_{\ge0}\), we have 
    \begin{align*}
        (x + y)^n = \sum_{k=0}^n \binom{n}{k} x^k y^{n - k} = \sum_{k=0}^n \binom{n}{k} y^k x^{n - k}.
    \end{align*}

    % \vspace{2em}
    \emph{Taylor series:} for all \(x \in \mathbb{R}\), we have
    \begin{align*}
        \mathrm{e}^x = \sum_{k=0}^{\infty} \frac{x^k}{k!}.
    \end{align*}

\end{frame}

%' ============================================================================================================================================================
\begin{frame}{Probability mass functions}

    % Let \(\mathcal{C} \subseteq \mathcal{D}\) be a set of possible values that \(X\) can obtain. 

    % \vspace{2em}
    The \emph{probability mass function (pmf)} of a discrete random variable \(X\) with support \(\mathcal{D}\) is a function \(f : \mathcal{D} \to [0,1]\) where 
    \begin{align*}
        f(x) = \mathrm{Pr}(X = x).
    \end{align*}
    In other words, it assigns a \textsl{probability} to each possible value of \(X\).

    \vspace{2em}
    If \(\mathcal{C} \subseteq \mathcal{D}\), we have 
    \begin{align*}
        \mathrm{Pr}(X \in \mathcal{C}) = \sum_{x \in \mathcal{C}} f(x).
    \end{align*}

    \vspace{2em}
    A valid pdf has the following proparties:
    \begin{itemize}
        \item \(f(x) \ge 0\) for all \(x \in \mathcal{D}\),
        \item \(f(x) = 0\) if \(x \not\in \mathcal{D}\),
        \item \(\sum_{x \in \mathcal{D}} f(x) = 1\).
    \end{itemize}

\end{frame}

% %' ============================================================================================================================================================
% \begin{frame}{Uniorm Distribution}

%     A discrete random variable \(X\) has a \emph{uniform distribution} with pdf
%     \begin{align*}
%         f(x;a,b) = \frac{1}{b - a + 1}
%     \end{align*}
%     for all \(x \in \mathcal{D} = \{a,a+1,\ldots,b-1,b\}\).

%     \vspace{2em}
%     It has the following properties:
%     \begin{itemize}
%         \item \(\displaystyle F(x) = \frac{x - a + 1}{b - a + 1}\) for all \(x \in \mathcal{D}\)
%         \item \(\displaystyle \mathrm{E}[X] = \frac{a + b}{2}\)
%         \item \(\displaystyle \mathrm{Var}[X] = \frac{(b-a+1)^2 - 1}{12}\)
%         \item \(\displaystyle \mathrm{M}_X(t) = \frac{\mathrm{e}^{at} - \mathrm{e}^{(b+1)t}}{(b-a+1)(1 - \mathrm{e}^t)}\)
%     \end{itemize}
%     % \begin{align*}
%     %     &\mathcal{D} = \{a,a+1,\ldots,b-1,b\} \\
%     %     &f(x;a,b) = \frac{1}{b-a+1} \\
%     %     &F(x;a,b) = \frac{x - a + 1}{b - a + 1} \\
%     %     &\mathrm{E}[X] = \frac{a + b}{2} \\
%     %     &\mathrm{Var}[X] = \frac{(b-a+1)^2 - 1}{12} \\
%     %     &\mathrm{M}_X(t) = \frac{\mathrm{e}^{at} - \mathrm{e}^{(b+1)t}}{(b-a+1)(1 - \mathrm{e}^t)}
%     % \end{align*}
%     % \begin{table}
%     %     \def\arraystretch{1.25}
%     %     \centering
%     %     \begin{tabular}[ht]{cccc} \hline
%     %         \(F(x)\) & \(\mathrm{E}[X]\) & \(\mathrm{Var}[X]\) & \(\mathrm{M}_X(t)\) \\\hline
%     %         \(\frac{x - a + 1}{b - a + 1}\) & \(\frac{a+b}{2}\) & \(\frac{(b-a+1)^2 - 1}{12}\) & \(\frac{\mathrm{e}^{at} - \mathrm{e}^{(b+1)t}}{(b-a+1)(1 - \mathrm{e}^t)}\)\\\hline
%     %     \end{tabular}
%     % \end{table}

% \end{frame}

%' ============================================================================================================================================================
\begin{frame}{Geometric Distribution}

    A random variable \(X\) has a \emph{geometric distribution} with pmf and cdf
    \begin{align*}
        f(x;p) &= \begin{cases}
            (1 - p)^x p & x = \{ 1, \infty ) \\
            \hfil 0 & \text{elsewhere}
        \end{cases} \\[1em]
        F(x;p) &= \begin{cases}
            \hfil 0 & x < 0 \\
            1 - (1 - p)^x & x = \{ 1, \infty )
        \end{cases}
    \end{align*}
    where \(p \in [0,1]\) is the \emph{probability of success}.
    \begin{itemize}
        \item \(X\) represents the number of Bernoulli trials needed before a success occurs. 
        \item \(X\) has the \emph{memoryless property}: \(\mathrm{Pr}(X > n \;|\; X > m) = \mathrm{Pr}(X > n - m)\).
    \end{itemize}

    \vspace{2em}
    We say \(X \sim \mathrm{Geo}(p)\), and we have:
    \begin{itemize}
        \item \(\mathrm{E}[X] = 1/p\)
        \item \(\mathrm{Var}[X] = (1 - p)/p^2\)
        \item \(\mathrm{M}_X(t) = p \mathrm{e}^t / \big( 1 - (1 - p)\mathrm{e}^t \big)\)
    \end{itemize}

\end{frame}

%' ============================================================================================================================================================
\begin{frame}{Binomial Distribution}

    A random variable \(X\) has a \emph{binomial distribution} with pmf and cdf 
    \begin{align*}
        f(x;n,p) &= \begin{cases}
            % \binom{n}{x} p^x (1 - p)^{n - x} & x \in \{ 0, n \} \\
            \begin{pmatrix}
                n \\ x
            \end{pmatrix} p^x (1 - p)^{n - x} & x \in \{ 0, n \} \\
                \hfil 0 & \text{elsewhere}
        \end{cases} \\[1em]
        F(x;n,p) &=  \begin{cases}
            \hfil 0 & x < 0 \\
            \displaystyle \sum_{k=0}^x f(k) & x \in \{0, n\} \\
            \hfil 1 & x > n
        \end{cases}
    \end{align*}
    where \(p \in [0,1]\) is the \emph{probability of success} and \(n \in \mathbb{N}\) is the \emph{number of trials.}
    \begin{itemize}
        \item \(X\) represents the number of successed oberved after \(n\) Bernoulli trials are conducted. 
    \end{itemize}

    \vspace{2em}
    We say \(X \sim \mathrm{Bin}(n,p)\), and we have:
    \begin{itemize}
        \item \(\mathrm{E}[X] = np\)
        \item \(\mathrm{Var}[X] = np(1 - p)\)
        \item \(\mathrm{M}_X(t) = \big( p(\mathrm{e}^t - 1) + 1 \big)^n\)
    \end{itemize}

\end{frame}

%' ============================================================================================================================================================
\begin{frame}{}

    

\end{frame}

\end{document}